% !TEX engine = xelatex

\documentclass[10pt,oneside,a4paper,landscape]{memoir}

\usepackage[lmargin=15mm,rmargin=15mm,bmargin=20mm,tmargin=20mm,bindingoffset=0mm,heightrounded]{geometry}

\usepackage[T1]{fontenc}
\usepackage [english]{babel}
\usepackage [autostyle, english = american]{csquotes}
\MakeOuterQuote{"}

\usepackage[absolute]{textpos}

\usepackage{graphicx}

\setsecnumformat{}

\newlength\tindent
\setlength{\tindent}{\parindent}
\setlength{\parindent}{0pt}
\renewcommand{\indent}{\hspace*{\tindent}}
\setlength{\parskip}{.5\baselineskip}%
\setlength{\parindent}{0pt}%

% Fonts

\usepackage{fontspec,xltxtra,xunicode}
\defaultfontfeatures{Mapping=tex-text}
\setsansfont[Scale=MatchLowercase,Mapping=tex-text]{Avenir-Medium}
\setmonofont[Scale=MatchLowercase]{Andale Mono}
\setmainfont[Ligatures=TeX,Scale=1]{Alegreya}

\newfontfamily\titlefont[Ligatures=TeX,Scale=1.9]{Almendra SC Bold}
\newfontfamily\sectionfont[Ligatures=TeX,Scale=1]{Amaranth Bold}
\newfontfamily\promofont[Ligatures=TeX,Scale=1.05]{Istok Web}
\newfontfamily\playfont[Ligatures=TeX,Scale=1]{Istok Web}
\newfontfamily\asidefont[Ligatures=TeX,Scale=1]{Istok Web}

%\newfontfamily\ornament[Ligatures=TeX,Scale=1]{Nymphette}

\usepackage{adforn}
\newcommand*\sep[0]{\adforn{73} }
\newcommand*\widesep[0]{\adforn{21} }

\usepackage[table]{xcolor}

\usepackage{multicol}
\usepackage{tabularx}
\usepackage{multirow}

\setlength\columnsep{10mm}

\usepackage{tikz}
\usetikzlibrary{decorations.text}

\usepackage{rotating}

\usepackage[absolute]{textpos}
\setlength{\TPHorizModule}{1mm}
\setlength{\TPVertModule}{1mm}

\definecolor{plainboxbg}{gray}{.95}
\definecolor{margintext}{rgb}{1,0,0}
\definecolor{offwhite}{HTML}{F1f1f1}

\usepackage{tcolorbox}
\tcbuselibrary{skins}
\tcbuselibrary{hooks}
\tcbuselibrary{raster}

\newtcolorbox{plainboxx}[1][]{fonttitle=\color{white}\headfont,arc=4mm,outer arc=1mm, colframe=black, colback=plainboxbg, #1}

\usepackage{hyperref}

% new commands
\newcommand*\sect[1]{\section{\sectionfont{#1}}}
\newcommand*\subsect[1]{\subsection{\sectionfont{#1}}}

\newcommand*\logo[1]{\begin{center}\titlefont{#1}\end{center}}
\newcommand*\promo[1]{\promofont{\large{#1}}}
\newcommand*\aside[1]{{\textit{\asidefont{#1}}}}
\newcommand*\play[1]{{\playfont{#1}}}

\usepackage{eso-pic}% http://ctan.org/pkg/eso-pic
\usepackage{graphicx}

\newcommand\AtPageUpperMyLeft[1]{\AtPageUpperLeft{%
 \put(\LenToUnit{.5cm},\LenToUnit{-.5cm}){#1}%
 }}%

\newcommand\AtPageLowerMyRight[1]{\AtPageUpperLeft{%
  \put(\LenToUnit{23cm},\LenToUnit{-14.5cm}){#1}%
  }}%

\graphicspath{ {images/} }

% this will put a framing picture in the upper left and lower right, image is from open clip art
%\AddToShipoutPictureBG{%
%  \AtPageUpperMyLeft{\raisebox{-\height}{\includegraphics[width=2.5in]{Anonymous-celtic-vine-corner-2400px.png}}}%
%  \AtPageLowerMyRight{\raisebox{-\height}{\includegraphics[width=2.5in,angle=180]{Anonymous-celtic-vine-corner-2400px.png}}}%
%}

\pagestyle{empty}

%%% BEGIN DOCUMENT

% The comments below here are bits of text that I liked but couldn't make fit

\begin{document}

\setlength\columnsep{10mm}
\begin{multicols}{2}

\setlength\columnsep{10mm}
  \begin{multicols}{2}

  {\logo{A Simple Solo Delve}}

  {\promo{\textbf{so·lo} [\textit{verb}] 1. to perform or accomplish something by oneself.}}

  Ready for adventure? You need a game system (like D\&D), some dice, a character, and a way to take notes.

  You'll play the game system you chose; this framework will play the GM's role.

  %And a willingness to step outside your usual "player" role and take some of the initiative.

  %\sect{Replacing the GM}

  Whenever the GM would do a GM thing -- set a scene, answer a question, act as an NPC -- ask the Oracle or roll on a random chart instead.

  When interpreting, trust your instincts and discard nonsense!

  %There are many Oracles, but here's one.

  %Frame a "yes" or "no" question and roll a d6. Add +1 if it is likely, -1 if it is unlikely.

  \normalsize
  \sect{The Oracle}
  \begin{tabularx}{\columnwidth}{XXXXXX}
    1-&2&3&4&5&6+\\
  No and & No & No but & Yes but & Yes & Yes and \\
  \end{tabularx}
  \normalsize

  An "and" answer is intensified. A "but" answer is twisted or weakened.

  \sect{Pushing Forward}

  When you're not sure what's next or you roll doubles, roll an Element and Event and interpret them as a surprise event.

  \sect{Goals}

  When you set a goal, explore 1d6 rooms before rolling again. Add +1 to each roll after the first; on a 7+ it's in the next room.

  \vfill\null
  \columnbreak

  \aside{I've made my hero using the rules in the handbook; it's time to roll a starting scene. I roll a d6 and get a 1.}

  \aside{lost in the desert \sep at sea \sep in a tavern \sep in a forest \sep in a tomb \sep in a fight!}

  \play{I'm lost in the desert after a bandit attack on my caravan.}

  \aside{Is there anything nearby? 6; yes, and... it's large. A pyramid? 5, yes.}

  \play{I head in, looking for resources.}

  \center \widesep

  \flushleft

  \aside{So what's this room like?}

  \aside {crumbling \sep dirty \sep disused \sep smooth \sep lavish \sep slippery}

  \aside{large \sep small \sep medium \sep cavern \sep great \sep chamber}

  %\aside{round \sep square \sep oval \sep elongated \sep rectangular \sep trapezoidal}

  \aside{sleep \sep eat \sep bodily function \sep prison \sep work \sep play }

  \aside{A crumbling (1), dirty (2) cavern (4) that's used for sleeping (4)? Sounds like a barracks.}

  \aside{[2-] Monster \sep [3] Trap \sep [4] Special \sep [6+] Empty}

  \aside{A monster (1), a hungry (1-4) spider (2-5)! Time to use the system mechanics. I look up a stat block, then roll to sneak by.}

  \aside{Failure! It's a fight! I turn to the "Combat" section of my rulebook...}

  \center \widesep

  \flushleft

  \vfill\null

  \end{multicols}

  \columnbreak

  \begin{multicols}{2}

    \aside{I've fought my way down, to a room with a Special feature (5).}

    \aside{A statue (1-2), of the Sun (6-1). Is it portable? (1) No, and it weighs a ton.}

    \play{Recklessly, I touch it.}

    \aside{stronger \sep weaker \sep injury \sep attack \sep asleep \sep teleport}

    \aside{Stronger (1). Is it temporary? (5) Yes. I check for a potion to use as the base.}

    \center \widesep

    \flushleft

    \sect{What Does It Do?}
    \footnotesize
    \begin{tabularx}{\columnwidth}{p{.05cm}|X|p{.05cm}|X|p{.05cm}|X}
    1 & Search & 2 & Steal & 3 & Fix \\
    4 & Hunt & 5 & Build & 6 & Kill \\
    \end{tabularx}
    \normalsize

    \sect{Why?}
    \footnotesize
    \begin{tabularx}{\columnwidth}{p{.05cm}|X|p{.05cm}|X|p{.05cm}|X}
    1 & Rebel & 2 & History & 3 & Desire \\
    4 & Honor & 5 & Mistake & 6 & Debt \\
    \end{tabularx}
    \normalsize

    \vfill\null
    \columnbreak

    \flushleft

    \aside{Time to move. How many exits? 1 (a d6, divided by 2, and rounded up).}

    \aside{north \sep south \sep east \sep west \sep up \sep down}

    \aside{straight \sep bend left \sep bend right \sep slope up \sep slope down \sep doubles back}

    \aside{Looks like I'm headed down (6), on a slope (5). Adventure awaits!}

    \center \widesep

    \flushleft

    \sect{What's the NPC Do?}
    \footnotesize
    \begin{tabularx}{\columnwidth}{p{.05cm}|X|p{.05cm}|X|p{.05cm}|X}
    1 & Indulges & 2 & Betrays & 3 & Fails \\
    4 & Helps & 5 & Reveals & 6 & Flees \\
    \end{tabularx}
    \normalsize

    \sect{Event}
    \footnotesize
    \begin{tabularx}{\columnwidth}{p{.05cm}|X|p{.05cm}|X|p{.05cm}|X}
    1 & Injury & 2 & Escalation & 3 & Backfire! \\
    4 & NPC & 5 & Obstacle & 6 & Reversal \\
    \end{tabularx}
    \normalsize

  \end{multicols}

  \sect{Room Contents}
  \footnotesize
  \begin{tabularx}{\columnwidth}{p{.05cm}|X|X|X|X|X|X}
   & 1 & 2 & 3 & 4 & 5 & 6 \\
  1 & Cage & Statue & Sack & Box & Draft & Body \\
  2 & Bucket & Perfume & Blood & Cauldron & Scrolls & Chest \\
  3 & Pedestal & Wet floor & Mural & Furniture & Golem & Mirror \\
  4 & Lake & Fountain & Jars & Bed & Camp & Rack \\
  5 & Forge & Pack & Books & Grate & Niche & Hole \\
  6 & Nest & Bones & Altar & Rubble & Chain & Toy \\
  \end{tabularx}
  \normalsize

  \sect{Elements (Literal or Figurative)}
  \footnotesize
  \begin{tabularx}{\columnwidth}{p{.05cm}|X|X|X|X|X|X}
   & 1 & 2 & 3 & 4 & 5 & 6 \\
  1 & Rider & Clover & Ship & House & Tree & Clouds \\
  2 & Snake & Coffin & Bouquet & Scythe & Whip & Birds \\
  3 & Child & Fox & Bear & Stars & Storks & Dog \\
  4 & Tower & Garden & Mountain & Crossroad & Mice & Heart \\
  5 & Ring & Book & Letter & Man & Woman & Lily \\
  6 & Sun & Moon & Key & Fish & Anchor & Cross \\
  \end{tabularx}
  \normalsize

  \begin{textblock*}{200mm}(165mm,190mm)
    {\sectionfont\center{\textit{\href{https://exposit.github.io/katarpgs/}{https://exposit.github.io/katarpgs}} \quad \textit{Last updated \textbf{\today}.}}}
  \end{textblock*}

  \begin{textblock*}{100mm}(180mm,4mm)
  \noindent \textit{\textcolor{black}{1-2: cruel \sep depraved \sep brave \sep hungry \sep miserable \sep insane \\ 3-4: degenerate \sep kind \sep obsessive \sep magical \sep superior \sep barbaric \\ 5-6: mutated \sep greedy \sep generous \sep hateful \sep fearful \sep trapped }}
  \end{textblock*}

  \begin{textblock*}{110mm}(10mm,192mm)
  \noindent \textit{\textcolor{black}{
  1-2: beast \sep sentient beast \sep goblinoid \sep elf \sep spider \sep aberration \\
  3-4: insect \sep humanoid \sep demihuman \sep automaton \sep elemental \sep magical beast \\
  5-6: plant \sep mindless undead \sep vermin \sep sentient undead \sep demon  \sep abomination }}
  \end{textblock*}

  \begin{textblock*}{200mm}(155mm,202mm)
  \noindent \textit{\textcolor{black}{
  These techniques are universal to soloing; you can find more at the \href{https://plus.google.com/communities/116965157741523529510}{Lone Wolf Google+ Group}}}
  \end{textblock*}


\end{multicols}

\end{document}
